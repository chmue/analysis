\begin{Sat}
  Zu jedem $x>0$ $(x\in\mb{R})$ und zu jedem $k\in\mb{N}$ gibt es eine reelle Zahl $y>0$ so dass $y^k=x$. In Zeichen:
  \[y=x^\frac{1}{k}, y=\sqrt[k]{x}\]
\end{Sat}
\begin{Bew}
  oBdA $x>1$ (sonst würden wir $\frac{1}{x}$ betrachten). wir konstruieren eine Intervallschachtelung $(I_n)$ so dass $\forall n a_n^k \geq x \geq b_n^k$
  \begin{equation*}
    I_1:=[1,x]\\
    I_{n+1} = \begin{cases}
      \left[ a_n, \frac{a_n+b_n}{2} \right] & \text{falls } x \leq \left( \frac{a_n+b_n}{2} \right)^k
      \left[ \frac{a_n+b_n}{2}, _n \right]
    \end{cases}\\
    \abs{I_n} = \frac{1}{2^{n-1}}\abs{I_1}
  \end{equation*}
  Intervallschachtelungsprinzip $\implies$ $\exists y\in\mb{R}$ s.d. $y\in I_n \forall n\in\mb{N}$ 
\end{Bew}
\begin{Sat}
  $y^k=x$
\end{Sat}
\begin{Bew}
   Man definiert $J_n=[a_n^k, b_n^k$. Wir wollen zeigen, dass $J_n$ eine Intervallschachtelung ist.
  \begin{itemize}
    \item $J_{n+1}\subset J_n$ weil $I_{n+1}\subset I_n$
    \item \[\abs{J_n} = b_n^k-a_n^k = \underbrace{\left( b_n - a_n \right)}_{\abs{I_n}} \underbrace{(b_n^{k-1}+b_n^{k-2}a_n + \cdots + a_n^{k-1})}_{\leq k b_1^{k-1}}\]
  \end{itemize}
  $\implies$ $\abs{J_n}\leq \abs{I_n}k k_1^{k-1}$.\\
  Sei $\varepsilon$ gegeben. Man wähle $N$ gross genug, so dass
  \[\abs{I_n}\leq\varepsilon'=\frac{\varepsilon}{kb_1^{k-1}}\implies\abs{J_n}\leq \varepsilon kb_1^{k-1}=\varepsilon\]
  Einerseits
  \[y\in\left[ a_n,b_n \right]\implies y^k\in\left[ a_n^k, b_n^k \right]=J_n\]
  Andererseits
  \[x\in J_n\forall n\in\mb{N}\]
  Intervallschachtelungsprinzip $\implies$ $x=y^k$
\end{Bew}
\subsection{Supremumseigenschaft, Vollständigkeit}
\begin{Def}
  $s\in\mb{R}$ heisst obere (untere) Schranke der Menge $M\subset \mb{R}$ falls $s\geq x$ ($s\leq x$) $\forall x\in M$.
\end{Def}
\begin{Def}
  $s\in\mb{R}$ ist das Supremum der Menge $M\subset\mb{R}$ falls es die kleinste obere Schranke ist. D.h.
  \begin{itemize}
    \item $s$ ist die obere Schranke
    \item falls $s'<s$, dass ist $s'$ keine obere Schranke.
  \end{itemize}
\end{Def}
\begin{Bsp}
  $M=]0,1[$. In diesem Fall $s=\sup M\not\in M$
\end{Bsp}
\begin{Bsp}
  $M=[0,1]$. $\sup M=1\in M$
\end{Bsp}
\begin{Def}
  $s\in\mb{R}$ heisst Infimum einer Menge $M$ ($s=\inf M$) falls $s$ die grösste obere Schranke ist.
\end{Def}
\begin{Def}
  Falls $s=\sup M\in M$, nennt man $s$ das Maximum von $M$. Kurz: $s=\max M$. Analog Minimum.
\end{Def}
\begin{Sat}
  Falls $M\subset \mb{R}$ nach oben (unten) beschränkst ist, dann existiert $\sup M$ ($\inf M$).
\end{Sat}
\begin{Bew}
  Wir konstruieren eine Intervallschachtelung $I_n$, so dass $b_n$ eine obere Schranke ist, und $a_n$ keine obere Schranke ist.
  \begin{itemize}
    \item $I_1=[a_1, b_1]$, wobei $b_1$ eine obere Schranke
    \item $a_1$ ist keine obere Schranke
  \end{itemize}
  Sei $I_n$ gegeben. 
  \begin{align*}
    I_{n+1} = \begin{cases}
      \left[ a_n,\frac{a_n+b_n}{2} \right]&\text{Falls $\frac{a_n+b_n}{2}$ eine obere Schranke ist-}\\
      \left[ \frac{a_n+b_n}{2}, b_n \right]&\text{sonst}
    \end{cases}
  \end{align*}
  Also, $\exists s\in I_n\forall n$
\end{Bew}
\begin{Sat}
  $s$ ist das Supremum von $M$
  \begin{itemize}
    \item Warum ist $s$ eine obere Schranke? \\
    Angenommen $\exists x\in M$ so dass $x>s$. Man wähle $\abs{I_n}<x-s$. Daraus folgt
    \begin{align*}
      x-s>b_n-a_n \geq b_n-s \implies x>b_n
    \end{align*}
    Widerspruch.
  \item Warum ist $s$ die kleinste obere Schranke?\\
    Angenommen $\exists s'<s$. Dann wähle $n'$ so dass $I_{n'} <s-s'$.
    \begin{align*}
      s-s'>b_{n'}-a_{n'}\geq s-a_{n'} \implies a_{n'}>s'
    \end{align*}
    Widerspruch.
  \end{itemize}
\end{Sat}
\begin{Lem}
  Jede nach oben (unten) beschränkte Menge $M\subseteq \mb{Z}$ besitzt das grösste (kleinste) Element.
\end{Lem}
\begin{Bew}
  oBdA betrachte nur nach unten beschränkte Mengen $M\subset N$. Angenommen $M$ hat kein kleinstes Element.
\end{Bew}
\begin{Sat}
  \begin{align*}
    \forall n M\cap \left\{ 1,\cdots,n \right\} = \null\\
    n=1\\
    M\cap\left\{ 1 \right\}\\
  \end{align*}
  Angenommen
  \begin{align*}
    M\cap\left\{ 1,\cdots,n \right\} = \null\\
    M\cap\left\{ 1,2,\cdots,n+1 \right\} = M\cap \left\{ 1,\cdots,n \right\}\cup M\cap\left\{ n+1 \right\}=\null\\
    \implies M\cap\mb{N}=\null
  \end{align*}
\end{Sat}
\begin{Sat}
  $\mb{Q}$ ist dich in $\mb{R}$, bzw. für beliebige zwei $x,y\in\mb{R}$, $y>x$, gibt es eine rationelle Zahl $q\in\mb{Q}$, so dass $x<q<y$.
\end{Sat}
\begin{Bew}
  Man wähle $n\in\mb{N}$ so dass $\frac{1}{n}<y-x$. Betrachte die Menge $A\subseteq\mb{Z}$, so dass $M\in A$ $\implies$ $M>nx$. Lemma $\implies$ $\exists m=\min A$.
  \begin{align*}
    x<\frac{m}{n}=\frac{m-1}{n}+\frac{1}{n}<x+y-x=y
  \end{align*}
  Also setze $q=\frac{m}{n}$
\end{Bew}
\subsection{Abzählbarkeit}
\begin{Def}
  Die Mengen $A$ \& $B$ sind \underline{gleichmächtig}, wenn es eine Bijektion $f:A\to B$ gibt. $A$ hat grässere Mächtigkeit als $B$, falls $B$ gleichmächtig wie eine Teilmenge von $A$ ist, aber $A$ zu keiner Teilmenge von $B$ gleichmächtig ist.
\end{Def}
\begin{Bsp}
  \begin{itemize}
    \item ${1,2}$ \& ${3,4}$ sind gleichächtig.
    \item ${1,2,\cdots,n}$ hat kleinere Mächtigkeit als ${1,2,\cdots,m}$, wenn $n<m$ ist.
  \end{itemize}
\end{Bsp}
\begin{Def}
  Eine Menge $A$ ist abzählbar, wenn es eine Bijektion zwischen $\mb{N}$ und $A$ gibt. D.h. $A=\left\{ a_1,a_2,\cdots,a_n,\cdots \right\}$.
\end{Def}
\begin{Lem}
  $\mb{Z}$ ist abzählbar
\end{Lem}
\begin{Bew}
  \begin{tabular}{c|cccccc}
    $\mb{N}$ & 1 & 2 & 3 & 4 & 5 & \ldots \\
    $\mb{Z}$ & 0 & 1 & -1 & 2 & -2 & \ldots
  \end{tabular}
  Formal:
  \begin{align*}
    f=\mb{N}\to \mb{Z}\\
    f(n)=\begin{cases}
      \frac{n}{2} & \text{wenn $n$ gerade}\\
      \frac{1-n}{2} & \text{wenn $n$ ungerade}
    \end{cases}
  \end{align*}
\end{Bew}
\begin{Sat}
  $\mb{Q}$ ist abzählbar
\end{Sat}
\begin{Bew}
  Sucht euch die Graphik auf Wikipedia oder sonstwo.
\end{Bew}
\begin{Sat}
  $\mb{R}$ ist nicht abzählbar.
\end{Sat}
