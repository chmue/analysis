%%% Local Variables: 
%%% mode: latex
%%% TeX-master: "analysis_revamp_test"
%%% End: 


%% some math testing
% start easy
% \begin{equation*}
%   a = b
% \end{equation*}

Sei $M \ne \emptyset$ eine Menge. Eine innere Verkn�pfung auf $M$ ist
eine \emph{Abbildung} $\vysmwhtcircle \mathcolon M \times M \mapsto M$.

$\forall a \leq 0 \comma a \in \BbbR$ und $\forall k \in \BbbN \setminus
\lbrace{}0\rbrace \comma \exists x > 0, x \in \BbbR$, so dass gilt
\begin{equation}
  \label{eq:1}
  x^k = a.
\end{equation}


%\begin{definition}
  $\forall q = \frac{m}{n} \in \BbbQ$ und $\forall q > 0$ definieren wir
%\end{definition}

Aus Gleichung \ref{eq:1} folgt xy.


\section{Die nat�rlichen Zahlen}

$\BbbN = \lbrace 1, 2, 3, \ldots , n, n+1, \ldots \rbrace$


\subsection{Vollst�ndige Induktion}

Wir haben einen Behauptung $B(n)$ �ber die nat�rlichen Zahlen. Die
Behauptung gilt als bewiesen, wenn \ldots

\begin{enumerate}
\item \ldots $B(1)$ richtig ist (Induktionsverankerung), und
\item \ldots $B(n) \rightarrow B(n+1)$ richtig ist (Induktionsschluss).
\end{enumerate}

\begin{Bem}[Zur Induktionsverankerung]
Falls die Induktionsverankerung aus Definition 1 nicht f�r $n = 1$
sondern f�r ein anderes $n \in \BbbN$ bewiesen werden kann, so gilt
die durch die vollst�ndige Induktion bewiesen Behauptung $\forall
m \geq n$.
\end{Bem}
