\documentclass[11pt, paper=a4]{scrartcl}

% XeLaTeX
\usepackage{xltxtra}

% crucial packages
% \usepackage[utf8]{inputenc}
% \usepackage[ngerman]{babel} % latex language package
\usepackage{polyglossia} % xelatex language package
\setmainlanguage[spelling=new,latesthyphen=false]{german}
\usepackage{xcolor}
\usepackage{graphicx}
\usepackage{tabularx}

% graphics


% math
\usepackage{amsmath} % basic ams package
% \usepackage{amsfonts} % font support
% \usepackage{amssymb}

%math-style=ISO,bold-style=tex

% theorem support (load before ref)
\usepackage{amsthm} % theorem-environments
\usepackage{thmtools} % extended theorem support

% ref support
\usepackage[xetex]{hyperref}
\usepackage{cleveref}

% unicode math support (load after all math packages!)
\usepackage{unicode-math}

% \KOMAoptions{DIV=14}

% Font Setup
\defaultfontfeatures{Scale=MatchUppercase}

\setmainfont[Path=C:/Windows/Fonts/,
UprightFont=LINLIBERTINE_RE-4.7.5.OTF,
BoldFont=LINLIBERTINE_BD-4.1.5.OTF,
ItalicFont=LINLIBERTINE_IT-4.2.6.OTF,
BoldItalicFont=LINLIBERTINE_BI-4.1.0.OTF,
Mapping=tex-text
]{Linux Libertine}

% \setmainfont[Mapping=tex-text]{XITS}

\setsansfont[Path=C:/Windows/Fonts/,
UprightFont=LinBiolinum_Re-0.6.4.otf,
BoldFont=LinBiolinum_Bd-0.5.5.otf,
ItalicFont=LinBiolinum_It-0.5.1.otf,
Mapping=tex-text,
Script=latn,
Language=DEU,
Numbers={Proportional}]{Linux Biolinum}



\setmathfont{XITS Math}



%= Titelseite ===========================================================================
\begin{document}
% \headheight15pt
\begin{titlepage}
\hfill
\vspace{20mm}
\pagenumbering{roman}
\begin{center}
{\LARGE \sffamily Analysis I - Vorlesungs-Script} \vskip 3em {\large Prof. Dr. Camillo De Lellis} \vskip 1.5em
{\large Basisjahr 10 Semester II}\vspace{30mm}\\
{\large {\bf Mitschrift:} \vspace{2mm}\\
Simon Hafner}\vspace{5mm}\\ %30mm
%{\large {\bf Graphics:} \vspace{2mm}\\
%Pirmin Weigele }\vspace{30mm}\\ %30mm
\author{Simon Hafner}

\end{center}
\vfill

\end{titlepage}


%= Inhaltsverzeichnis ==========================================================================
% \lhead{}
% \rhead{}
\tableofcontents
\newpage
\pagenumbering{arabic}
\setcounter{page}{1}

%= Vorlesung-Skripts ==========================================================================
% \cfoot{\thepage}
% \fancyhead[L]{\nouppercase{\leftmark}}
\newpage

%= Analysis I & & II ==========================================================================

%Analysis I
%% some math testing
% start easy
% \begin{equation*}
%   a = b
% \end{equation*}

Sei $M \ne \emptyset$ eine Menge. Eine innere Verknüpfung auf $M$ ist
eine \emph{Abbildung} $\vysmwhtcircle \mathcolon M \times M \mapsto M$.

$\forall a \leq 0 \comma a \in \BbbR$ und $\forall k \in \BbbN \setminus
\lbrace{}0\rbrace \comma \exists x > 0, x \in \BbbR$, so dass gilt
\begin{equation}
  \label{eq:1}
  x^k = a.
\end{equation}


%\begin{definition}
  $\forall q = \frac{m}{n} \in \BbbQ$ und $\forall q > 0$ definieren wir
%\end{definition}

Aus Gleichung \ref{eq:1} folgt xy.

\newpage

%= Stichwortverzeichnis ======================================================================
% \rhead{}
\addcontentsline{toc}{section}{Stichwortverzeichnis}
% \printindex

\end{document}
